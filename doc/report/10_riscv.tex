\section{RISCV}

RISC-V is an open standard instruction set architecture (ISA) based in the reduced instruction set computer (RISC) architecture. This architecture is based on a small and higly optimized instructions in contrast with other types of architectures like the complex instruction set computer (CISC). The RISC-V ISA does not require fees to use and several companies and projects  are considering this architecture to developt their products. In addition, open source operating systems with RISC-V support are available and the instruction set is supported in several popular software toolchains. Load-store architecture and IEEE-754 floating point instructions. 

\subsection{History}

Prof. Krste Asanović and graduate students Yunsup Lee and Andrew Waterman started the RISC-V instruction set in May 2010 as part of the Parallel Computing Laboratory (Par Lab) at UC Berkeley, California.  The initial porpouse of RISC-V was to offer a open source hardware that could be used for academic porpouse and it could be deployable in any hardware or software design without royalties.\\

The architecture has as a precedent the DLX MIPS isntruction set by David Patterson. David Patterson joined the project and wasthe originator of the Berkeley RISC. RISC-V is the fifth generation of a serie of cooperative RISC-based research projects. The authors and their institutions originally sourced the ISA documents and several CPU designs which would allow derivative work to be reather open and free or close and propety. The ISA specification was published in 2011  with all rights reserved. \\

RISC-V fundation was created to own, maintain, and publish intellectual property related to RISC-V's definition. Comercial users needed a stable ISA to develop products that would be used for years. For this reason, the authors and owners had to give their rights to the fundation. The fundation moved from US to Switzerland concerning the US trade conditions in 2019. Its named changed to RISC-V International  and since then they have freely published the documents defining RISC-V and permits unrestricted use of the ISA for design of software and hardware. However, changes can only be accepted by the members of the fundation. 

\subsection{ISA base and extensions} 
One of the more interesting carachteritics abour RISC-V is the modularity and this is reflected in the ISA extensions that consit in alternative base  and optional extensions. The base can implement a simplified general-purpose computer including compilers. It specifies: 

\begin{itemize}
	\item Instructions and their encodings
	\item Control flow
	\item Registers and their size
	\item Memory and addressing
	\item Logic manipulation
\end{itemize}

In the following table, the ISA modules are listed and described:

\begin{table}[H]
\centering
\begin{tabular}{|c|c|c|c|}
\hline
\textbf{Name} & \textbf{Description} & \textbf{Status}  & \textbf{Ins. Count}  \\ \hline

RVWMO & Weak Memory Ordering & Ratified & \\ \hline
RV32I & Base Integer Instruction set (32 bits) & Ratified & 49 \\ \hline
RV32E & Base Integer Instruction set embedded & Open & 49 \\ \hline
RV64I & Base Integer Instruction set (64 bits) & Ratified & 14 \\ \hline
RV128I & Base Integer Instruction set (128 bits) & Open & 14 \\ \hline

\end{tabular}
\caption{RISC-V base ISA modules.} 
\end{table}


\begin{table}[H]
\centering
\begin{tabular}{|c|c|c|c|}
\hline
\textbf{Name} & \textbf{Description} & \textbf{Status}  & \textbf{Ins. Count}  \\ \hline

M & Multiplication Division & Ratified & 8\\ \hline
A & Atomic Instruction & Ratified & 11 \\ \hline
F & Single precision FP & Ratified & 25 \\ \hline
D & Double precision FP & Ratified & 25 \\ \hline
Q & Quad precision FP & Ratified & 27 \\ \hline
C & Compressed Instruction & Ratified & 36 \\ \hline
Zicsr & Control and Status Register (CSR) & Ratified &  \\ \hline
Zifencei & Instruction-Fetch Fence & Ratified & \\ \hline
\end{tabular}
\caption{RISC-V popular ISA extensions.} 
\end{table}

Required ISA modules it is a characteristic to evaluate possible RISC-V cores depending on the functionalitis the developers want to offer. 

\subsection{Registers}
RISC-V cores tipically have 32 integer registers and 32 floating point registers when this extension is implemented. Each register has its functionality that is a consense of the RISC-V developers. This is the method to keep information consistent and meaningful. It is critical to 


\begin{table}[H]
\centering
\begin{tabular}{|c|c|c|c|}
\hline
\textbf{Register name} & \textbf{Symbolic name} & \textbf{Description}  & \textbf{Ins. Count}  \\ \hline

M & Multiplication Division & Ratified & 8\\ \hline
A & Atomic Instruction & Ratified & 11 \\ \hline
F & Single precision FP & Ratified & 25 \\ \hline
D & Double precision FP & Ratified & 25 \\ \hline
Q & Quad precision FP & Ratified & 27 \\ \hline
C & Compressed Instruction & Ratified & 36 \\ \hline
Zicsr & Control and Status Register (CSR) & Ratified &  \\ \hline
Zifencei & Instruction-Fetch Fence & Ratified & \\ \hline
\end{tabular}
\caption{RISC-V popular ISA extensions.} 
\end{table}



