\section{Conclusions}
During this project, we ended up following a different path than the original we had planned, but this does not mean that we could not archive the main topics that we wanted to cover.
In other words, we could develop a project that involved the main topics, objectives: RISC-V, multicore and boot. 

We adopted an incremental strategy, which means that we performed little incremental steps on order to archive the set objectives. We started with a simple and undestandable RISC-V processors and we improved it.
First, we made it more modular.
Then, we implemented a memory \textit{referee} and made the processor multicore. 
And finally, we connected the processor \gls{rtl} to the \gls{fpga}'s \gls{ddr} implementing the corresponding logic.

During this process, we learnt to do an \gls{fpga} project from scratch, creating a functinal block design in Vivado, and understanding and adapting the constraint file necessary for the implementation step.

We could also use different Xilinx's \glspl{ip} from the catalog and learn how to use Vivado, an industry reference software. 


In conclusion, we learnt a lot from this project: RISC-V \gls{isa}, \gls{rtl} implementation with System Verilog, Vivado usage, validation in simulation and on the \gls{fpga} with \gls{ila}, and Xilinx's \glspl{ip}.

