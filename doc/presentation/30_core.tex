\begin{frame}{Core - Selection criteria}
  \begin{itemize}
    \item Multicore capable design.
    \item Linux bootable.
    \item Open/accessible design.
    \item Fit resource constraints of our \gls{fpga}. 
    \item Maximize compatibility with our \gls{fpga}.
  \end{itemize}
\end{frame}

\begin{frame}{Core - First try: Ariane}
\begin{columns}[T]
  \begin{column}{0.5\textwidth} % Left column and width

\begin{itemize}
    \item Ariane is a 6-stage in-order CPU.
    \item Implements privilege levels M, S, U to fully support a Unix-like operating system.
    \item Multicore via OpenPiton.
\end{itemize}
\end{column}
\begin{column}{.5\textwidth} % Right column and width

\begin{figure}[!ht]
    \includegraphics[width=1.1\linewidth]{images/ariane-diagram.png}
\end{figure}
\end{column}
\end{columns}
\end{frame}

\begin{frame}{Core - First try: Ariane}
    \begin{itemize}
            \item Boots Linux.
            \item Multi-Core capable.
            \item Open design.
    \end{itemize}

    \begin{alertblock}{Issue}
            Too complex and too much effort to port the design to be compatible with our \gls{fpga}.	
    \end{alertblock}
\end{frame}

\begin{frame}{Core - Second try: Rocket Chip}
\begin{columns}[T]
  \begin{column}{0.5\textwidth} % Left column and width
Rocket Chip is a SoC generator for Rocket Cores
\begin{itemize}
    \item Developed in Berkley, EEUU
    \item Hardware generation is done using Chisel
    \item Rocket core: 
    \begin{itemize}
        \item In-order scalar processor with 5-stage pipeline
        \item RV64G variant of the RISC-V ISA
    \end{itemize}
\end{itemize}
\end{column}
\begin{column}{.5\textwidth} % Right column and width

\begin{figure}[!ht]
    \includegraphics[width=1\linewidth]{images/rocketchip-diagram.png}
\end{figure}
\end{column}
\end{columns}
\end{frame}

\begin{frame}{Core - Second try: Rocket Chip}
  \begin{itemize}
    \item Multi-core system capable with option to add coherent memory systems.
    \item Boots Linux.
    \item Open source.
  \end{itemize}

  \begin{alertblock}{Issue}
    Scala based configuration. Lack of documentation. High \gls{fpga} resource usage. Not clear on how to generate a bitstream for our \gls{fpga}.
  \end{alertblock}
\end{frame}


\begin{frame}{Core - Final selection: Darkriscv}
  \begin{itemize}
    \item Low resource utilisation.
    \item Simple design.
    \begin{itemize}
      \item Implements RISC-V 32 bit: \textbf{E} and \textbf{I} extensions.
      \item \gls{rtl} designed in 3 files. 
    \end{itemize}
    \item Tested on similar \gls{fpga} boards.
  \end{itemize}
  \begin{exampleblock}{Approach}
    Add multicore support and boot parallel programs.
  \end{exampleblock}
\end{frame}

\begin{frame}{Core - Darkriscv}
  Deployment process:
 \begin{center}
    \begin{tikzpicture}[node distance=0.4cm]
      \node (prj) [stylepop] {Create Vivado project};
      \node (const) [stylepop, right=of prj ] {Design constraints file};
      \node (blk) [stylepop, right=of const] {Block design};
      \node (bit) [stylepop, right=of blk] {Generate bitstream};

      \draw [arrow] (prj) -- (const);
      \draw [arrow] (const) -- (blk);
      \draw [arrow] (blk) -- (bit);
    \end{tikzpicture}
  \end{center}
\end{frame}

\begin{frame}{Core - Darkriscv block design}
  Block design:
  \begin{itemize}
    \item Map the \gls{fpga} clock to Xilinx clock IP to manage the core clock.
    \item Bind reset button to core reset.
    \item Bind core leds to physical leds.
  \end{itemize}
 \begin{center}
\begin{figure}[!ht]
    \includegraphics[width=1\linewidth]{images/block-design.png}
\end{figure}
  \end{center}
\end{frame}