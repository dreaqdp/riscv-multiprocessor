\documentclass[xcolor=table]{beamer}
\usepackage[utf8]{inputenc}
\usepackage[spanish]{babel}
\usepackage{listings}
\usepackage{graphicx}
\usepackage{hyperref}
\usepackage[acronym]{glossaries}
\usepackage{appendixnumberbeamer}
\usepackage{caption} % captionof for figures

\usetheme{Dresden}
\usecolortheme{lily}

\newacronym{hpc}{HPC}{High Performance Computing}
\newacronym{isa}{ISA}{Instruction Set Architecture}
\newacronym{fpga}{FPGA}{Field Programmable Gate Array}
\newacronym{lut}{LUT}{Lock-Up Table}
\newacronym{dsp}{DSP}{Digital Signal Processors}
\newacronym{tlp}{TLP}{Transaction Layer Packet}
\newacronym{axi}{AXI-ST}{AXI4-Stream}
\newacronym{ip}{IP}{Intellectual Property}
\newacronym{vpu}{VPU}{Vector Processor Unit}
\newacronym{rtl}{RTL}{Register-Transfer Level}
\newacronym{dw}{Dword}{Double word}
\newacronym{hdl}{HDL}{Hardware Description Language}


\newacronym{epi}{EPI}{European Processor Iniciative}

\newacronym{io}{IO}{input/ouput}



\title{RISCV processor in an \acrshort{fpga}}
\author{Jose Estragues \\ Andrea Querol \\ Guillem Ramírez \\ Joan Vinyals \\ Pablo Vizcaino}
\date{May 2021}
\institute[FIB, UPC]{Facultat d'Informàtica de Barcelona \\ Universitat Politècnica de Catalunya - BarcelonaTech \and Barcelona Supercomputing Center}

\setbeamertemplate{page number in head/foot}[totalframenumber]

\AtBeginSection[]
{
  \begin{frame}<beamer>{Outline}
    \tableofcontents[currentsection]
  \end{frame}
}

\newcommand{\aqnote}[1]{ {\color{violet}\textbf{AQ:} #1 } }
\newcommand{\gnunote}[1]{ {\color{blue}\textbf{GNU:} #1 } }

\begin{document}

\begin{frame}
\maketitle
\end{frame}


\begin{frame}{}
    \tableofcontents
\end{frame}

\section{RISCV}
\begin{frame}{RISC-V introduction}
  \begin{itemize}
     \item Open standard instruction set architecture (ISA)
     \item Reduced instruction set computer (RISC)
     \item Designed by University of California, Berkeley. 
     \item Industry relevant:
       \begin{itemize}
         \item European Processor Initiative (EPI): RISC-V based european processor.
       \end{itemize}
  \end{itemize}
\end{frame}

\begin{frame}{RISC-V ISA details}
  \begin{itemize}
     \item Implements privileged and unprivileged ISA.
     \item 32 general purpose registers (16 in embedded)
     \item Minimal instruction set required. Extensions provide additional functionality:
       \begin{itemize}
         \item \textbf{M:} Multiplication
         \item \textbf{A:} Atomics
         \item \textbf{F, D, Q:} Floating point in 32, 64 and 128 bit respectively.
         \item \textbf{Zicsr:} Control and status register support.
         \item \textbf{Zifencei:} Load/Store fence
         \item \textbf{C:} Compressed instructions.
       \end{itemize}
  \end{itemize}
\end{frame}

\section{\acrshort{fpga}}
\begin{frame}{\acrshort{fpga} introduction}
    \begin{itemize}
        \item \acrfull{fpga}
        \item Xilinx Spartan 7 \cite{sp701} \cite{fpga_resources}
            \begin{itemize}
                \item x \gls{lut}
                \item \aqnote{posar resources (?) justificacio de decisions de disseny mes endavant}
            \end{itemize}
    \end{itemize}
\end{frame}

\section{Core}
\begin{frame}{Core - Selection criteria}
  \begin{itemize}
    \item Multicore capable design.
    \item Linux bootable.
    \item Open/accessible design.
    \item Fit resource constraints of our fpga. 
    \item Maximize compatibility with our fpga.
  \end{itemize}
\end{frame}

\section*{}

\begin{frame}[allowframebreaks]
        \frametitle{References}
\bibliographystyle{unsrt}
\bibliography{99_ref}
\end{frame}


\begin{frame}{}
    \centering
    \Large Thanks for your attention.
\end{frame}

\end{document}
